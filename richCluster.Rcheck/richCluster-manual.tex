\nonstopmode{}
\documentclass[a4paper]{book}
\usepackage[times,inconsolata,hyper]{Rd}
\usepackage{makeidx}
\makeatletter\@ifl@t@r\fmtversion{2018/04/01}{}{\usepackage[utf8]{inputenc}}\makeatother
% \usepackage{graphicx} % @USE GRAPHICX@
\makeindex{}
\begin{document}
\chapter*{}
\begin{center}
{\textbf{\huge Package `richCluster'}}
\par\bigskip{\large \today}
\end{center}
\ifthenelse{\boolean{Rd@use@hyper}}{\hypersetup{pdftitle = {richCluster: Fast, Robust Clustering Algorithms for Gene Enrichment Data}}}{}
\ifthenelse{\boolean{Rd@use@hyper}}{\hypersetup{pdfauthor = {Junguk Hur; Sarah Hong; Jane Kim}}}{}
\begin{description}
\raggedright{}
\item[Type]\AsIs{Package}
\item[Title]\AsIs{Fast, Robust Clustering Algorithms for Gene Enrichment Data}
\item[Version]\AsIs{1.0.0}
\item[Date]\AsIs{2025-09-20}
\item[Maintainer]\AsIs{Junguk Hur }\email{hurlabshared@gmail.com}\AsIs{}
\item[Description]\AsIs{The richCluster package contains a fast C++ agglomerative
hierarchical clustering algorithm packaged into easily callable R
functions, designed to help cluster biological 'terms' based on how
similar of genes are expressed in their activation.}
\item[License]\AsIs{MIT + file LICENSE}
\item[Depends]\AsIs{R (>= 3.5.0)}
\item[Imports]\AsIs{dplyr, fields, heatmaply, igraph, iheatmapr, magrittr,
networkD3, plotly, Rcpp (>= 1.0.14), stats, tidyr, viridis}
\item[Suggests]\AsIs{devtools, knitr, rmarkdown, roxygen2, testthat}
\item[LinkingTo]\AsIs{Rcpp}
\item[VignetteBuilder]\AsIs{knitr}
\item[Encoding]\AsIs{UTF-8}
\item[RoxygenNote]\AsIs{7.3.3}
\item[NeedsCompilation]\AsIs{yes}
\item[Author]\AsIs{Junguk Hur [aut, cre] (ORCID: <}\url{https://orcid.org/0000-0002-0736-2149}\AsIs{>),
Sarah Hong [aut],
Jane Kim [aut]}
\item[Archs]\AsIs{x64}
\end{description}
\Rdcontents{Contents}
\HeaderA{richCluster-package}{richCluster: clustering and visualization utilities}{richCluster.Rdash.package}
\aliasA{richCluster}{richCluster-package}{richCluster}
\keyword{internal}{richCluster-package}
%
\begin{Description}
Tools for clustering enriched terms, building correlation networks,
and producing interactive heatmaps and network views.
\end{Description}
%
\begin{Author}
\strong{Maintainer}: Junguk Hur \email{junguk.hur@med.und.edu}

Authors:
\begin{itemize}

\item{} Sarah Hong \email{ssh2198@columbia.edu}
\item{} Jane Kim \email{j6kim2026@chadwickschool.org}

\end{itemize}


\end{Author}
\HeaderA{cluster}{Cluster Terms from Enrichment Results}{cluster}
%
\begin{Description}
This function performs clustering on enrichment results by integrating
gene similarity scores and various clustering strategies.
\end{Description}
%
\begin{Usage}
\begin{verbatim}
cluster(
  enrichment_results,
  df_names = NULL,
  min_terms = 5,
  min_value = 0.1,
  distance_metric = "kappa",
  distance_cutoff = 0.5,
  linkage_method = "average",
  linkage_cutoff = 0.5
)
\end{verbatim}
\end{Usage}
%
\begin{Arguments}
\begin{ldescription}
\item[\code{enrichment\_results}] A list of dataframes, each containing enrichment results.
Each dataframe should include at least the columns 'Term', 'GeneID', and 'Padj'.

\item[\code{df\_names}] Optional, a character vector of names for the enrichment result dataframes. Must
match the length of `enrichment\_results`. Default is `NULL`.

\item[\code{min\_terms}] Minimum number of terms each final cluster must include

\item[\code{min\_value}] Minimum 'Pvalue' a term must have in order to be counted in final clustering

\item[\code{distance\_metric}] A string specifying the distance metric to use (e.g., "kappa").

\item[\code{distance\_cutoff}] A numeric value for the distance cutoff (0 < cutoff <= 1).

\item[\code{linkage\_method}] A string specifying the linkage method to use
(e.g., "average"). Supported options are "single", "complete",
"average", and "ward".

\item[\code{linkage\_cutoff}] A numeric value between 0 and 1 for the membership cutoff.
\end{ldescription}
\end{Arguments}
%
\begin{Value}
A named list containing:
- `distance\_matrix`: The distance matrix used in clustering.
- `clusters`: The final clusters.
- `df\_list`: The original list of enrichment result dataframes.
- `merged\_df`: The merged dataframe containing combined results.
- `cluster\_options`: A list of clustering parameters used in the analysis.
- `df\_names` (optional): The names of the input dataframes if provided.
\end{Value}
\HeaderA{cluster\_bar}{Cluster-level Bar Plot of Enrichment Significance}{cluster.Rul.bar}
%
\begin{Description}
Generates a horizontal bar plot showing average enrichment significance
for each cluster, across one or more enrichment datasets.
\end{Description}
%
\begin{Usage}
\begin{verbatim}
cluster_bar(cluster_result, clusters = NULL, value_type = "Padj", title = NULL)
\end{verbatim}
\end{Usage}
%
\begin{Arguments}
\begin{ldescription}
\item[\code{cluster\_result}] A result list returned by \code{\LinkA{cluster}{cluster}}.

\item[\code{clusters}] Optional numeric vector of cluster IDs to include. Defaults to all clusters.

\item[\code{value\_type}] The column name to use for enrichment significance ("Padj" or "Pvalue").

\item[\code{title}] Optional plot title. If NULL, a default will be generated.
\end{ldescription}
\end{Arguments}
%
\begin{Value}
A \code{plotly} object representing the bar plot.
\end{Value}
%
\begin{Examples}
\begin{ExampleCode}
## Not run: 
cbar <- cluster_bar(cluster_result)
cbar

## End(Not run)
\end{ExampleCode}
\end{Examples}
\HeaderA{cluster\_correlation\_hmap}{Create a Correlation Heatmap for a Specific Cluster}{cluster.Rul.correlation.Rul.hmap}
%
\begin{Description}
This function generates a correlation heatmap for a specific cluster based on the provided distance matrix.
\end{Description}
%
\begin{Usage}
\begin{verbatim}
cluster_correlation_hmap(
  final_clusters,
  distance_matrix,
  cluster_number,
  merged_df
)
\end{verbatim}
\end{Usage}
%
\begin{Arguments}
\begin{ldescription}
\item[\code{final\_clusters}] A dataframe containing the final cluster data.

\item[\code{distance\_matrix}] A matrix representing the distances between terms.

\item[\code{cluster\_number}] An integer specifying the cluster number to visualize.

\item[\code{merged\_df}] A dataframe with all terms used to map term indices to names.
\end{ldescription}
\end{Arguments}
%
\begin{Value}
An interactive heatmaply heatmap.
\end{Value}
\HeaderA{cluster\_dot}{Cluster-level Dot Plot of Enrichment Significance}{cluster.Rul.dot}
%
\begin{Description}
Creates a dot plot summarizing cluster-level enrichment across datasets.
Each point represents a cluster, with its size proportional to the number
of terms and its x-position reflecting average significance (e.g., Padj or Pvalue).
\end{Description}
%
\begin{Usage}
\begin{verbatim}
cluster_dot(cluster_result, clusters = NULL, value_type = "Padj", title = NULL)
\end{verbatim}
\end{Usage}
%
\begin{Arguments}
\begin{ldescription}
\item[\code{cluster\_result}] A result list returned from \code{\LinkA{cluster}{cluster}}.

\item[\code{clusters}] Optional numeric vector of cluster IDs to include. Defaults to all clusters.

\item[\code{value\_type}] The name of the value column to visualize (e.g., "Padj" or "Pvalue").

\item[\code{title}] Optional title for the plot. If NULL, a default title is generated.
\end{ldescription}
\end{Arguments}
%
\begin{Value}
A \code{plotly} object representing the dot plot.
\end{Value}
%
\begin{Examples}
\begin{ExampleCode}
## Not run: 
cdot <- cluster_dot(cluster_result)
cdot

## End(Not run)
\end{ExampleCode}
\end{Examples}
\HeaderA{cluster\_hmap}{Create a Heatmap of Clustered Enrichment Results}{cluster.Rul.hmap}
%
\begin{Description}
Generates an interactive heatmap from the given clustering results,
visualizing -log10(Padj) values for each cluster. The function aggregates
values per cluster and assigns representative terms as row names.
\end{Description}
%
\begin{Usage}
\begin{verbatim}
cluster_hmap(
  cluster_result,
  clusters = NULL,
  value_type = "Padj",
  aggr_type = mean
)
\end{verbatim}
\end{Usage}
%
\begin{Arguments}
\begin{ldescription}
\item[\code{cluster\_result}] A list containing a data frame (`cluster\_df`) with clustering results.
The data frame must contain at least the columns `Cluster`, `Term`, and `value\_type\_*` values.

\item[\code{clusters}] Optional. A numeric or character vector specifying the clusters to include.
If NULL (default), all clusters are included.

\item[\code{value\_type}] A character string specifying the column name prefix for values to display in hmap cells.
Defaults to `"Padj"`.

\item[\code{aggr\_type}] A function used to aggregate values across clusters (e.g., `mean` or `median`).
Defaults to `mean`.
\end{ldescription}
\end{Arguments}
%
\begin{Details}
The function processes the given cluster data frame (`cluster\_df`),
aggregating the `value\_type\_*` values per cluster using the specified `aggr\_type` function.
The -log10 transformation is applied, and infinite values are replaced with 0.

Representative terms are selected by choosing the term with the lowest
`value\_type` in each cluster.

The final heatmap is generated using `heatmaply::heatmaply()`, with
an interactive `plotly` visualization.
\end{Details}
%
\begin{Value}
An interactive heatmap object (`plotly`), displaying the -log10(Padj) values
across clusters, with representative terms as row labels.
\end{Value}
\HeaderA{cluster\_network}{Create a Network Graph for a Specific Cluster}{cluster.Rul.network}
%
\begin{Description}
This function generates a network graph for a specific cluster based on the provided distance matrix.
The opacity and length of the edges correspond to the given distance\_metric (eg, kappa) score similarity 
between terms, which is based on shared gene content.
\end{Description}
%
\begin{Usage}
\begin{verbatim}
cluster_network(final_clusters, distance_matrix, cluster_number, merged_df)
\end{verbatim}
\end{Usage}
%
\begin{Arguments}
\begin{ldescription}
\item[\code{final\_clusters}] A dataframe containing the final cluster data.

\item[\code{distance\_matrix}] A matrix representing the distances between terms.

\item[\code{cluster\_number}] An integer specifying the cluster number to visualize.

\item[\code{merged\_df}] A dataframe with all terms used to map term indices to names.
\end{ldescription}
\end{Arguments}
%
\begin{Value}
An interactive networkD3 network graph.
\end{Value}
\HeaderA{compare\_network\_graphs\_plotly}{Compare Network Graphs using Plotly}{compare.Rul.network.Rul.graphs.Rul.plotly}
%
\begin{Description}
This function creates a side-by-side comparison of network graphs for a single
cluster using different p-value types.
\end{Description}
%
\begin{Usage}
\begin{verbatim}
compare_network_graphs_plotly(cluster_result, cluster_num, pval_names)
\end{verbatim}
\end{Usage}
%
\begin{Arguments}
\begin{ldescription}
\item[\code{cluster\_result}] The result from the clustering function.

\item[\code{cluster\_num}] The cluster number to plot.

\item[\code{pval\_names}] A list of p-value names to compare.
\end{ldescription}
\end{Arguments}
%
\begin{Value}
A plotly object.
\end{Value}
\HeaderA{david\_cluster}{Cluster Terms using DAVID's method}{david.Rul.cluster}
%
\begin{Description}
This function performs clustering on enrichment results using an algorithm
inspired by DAVID's functional clustering method.
\end{Description}
%
\begin{Usage}
\begin{verbatim}
david_cluster(
  enrichment_results,
  df_names = NULL,
  similarity_threshold = 0.5,
  initial_group_membership = 3,
  final_group_membership = 3,
  multiple_linkage_threshold = 0.5
)
\end{verbatim}
\end{Usage}
%
\begin{Arguments}
\begin{ldescription}
\item[\code{enrichment\_results}] A list of dataframes, each containing enrichment results.
Each dataframe should include at least the columns 'Term', 'GeneID', and 'Padj'.

\item[\code{df\_names}] Optional, a character vector of names for the enrichment result dataframes. Must
match the length of `enrichment\_results`. Default is `NULL`.

\item[\code{similarity\_threshold}] A numeric value for the kappa score cutoff (0 < cutoff <= 1).

\item[\code{initial\_group\_membership}] Minimum number of terms to form an initial seed group.

\item[\code{final\_group\_membership}] Minimum number of terms for a final cluster.

\item[\code{multiple\_linkage\_threshold}] A numeric value for the merging threshold.
\end{ldescription}
\end{Arguments}
%
\begin{Value}
A named list containing the clustering results.
\end{Value}
\HeaderA{export\_df}{Export Cluster Result as Dataframe}{export.Rul.df}
%
\begin{Description}
Returns a comprehensive dataframe containing all the different terms in all clusters.
\end{Description}
%
\begin{Usage}
\begin{verbatim}
export_df(cluster_result)
\end{verbatim}
\end{Usage}
%
\begin{Arguments}
\begin{ldescription}
\item[\code{cluster\_result}] The cluster\_result object from cluster()
\end{ldescription}
\end{Arguments}
%
\begin{Value}
A data.frame view of the clustering
\end{Value}
\HeaderA{filter\_clusters}{Filter Clusters by Number of Terms}{filter.Rul.clusters}
%
\begin{Description}
Filters the full list of clusters by keeping only those with greater
than or equal to min\_terms \# of terms.
\end{Description}
%
\begin{Usage}
\begin{verbatim}
filter_clusters(all_clusters, min_terms)
\end{verbatim}
\end{Usage}
%
\begin{Arguments}
\begin{ldescription}
\item[\code{all\_clusters}] A dataframe containing the merged seeds with column named `ClusterIndices`.

\item[\code{min\_terms}] An integer specifying the minimum number of terms required in a cluster.
\end{ldescription}
\end{Arguments}
%
\begin{Value}
The filtered data frame with clusters filtered to include only those with at least `min\_terms` terms.
\end{Value}
\HeaderA{format\_colnames}{Format Column Names for Merging}{format.Rul.colnames}
%
\begin{Description}
This function maps a vector of column names to standardized names
for "GeneID", "Pvalue", and "Padj" based on known variations.
\end{Description}
%
\begin{Usage}
\begin{verbatim}
format_colnames(colnames)
\end{verbatim}
\end{Usage}
%
\begin{Arguments}
\begin{ldescription}
\item[\code{colnames}] A character vector of column names to be standardized.
\end{ldescription}
\end{Arguments}
%
\begin{Value}
A character vector of standardized column names.
\end{Value}
\HeaderA{full\_network}{Create a Network Graph for the Entire Distance Matrix}{full.Rul.network}
%
\begin{Description}
This function generates a network graph for the entire distance matrix.
\end{Description}
%
\begin{Usage}
\begin{verbatim}
full_network(cluster_result)
\end{verbatim}
\end{Usage}
%
\begin{Arguments}
\begin{ldescription}
\item[\code{cluster\_result}] Cluster result named list from richCluster::cluster()
\end{ldescription}
\end{Arguments}
%
\begin{Value}
An interactive networkD3 network graph.
\end{Value}
\HeaderA{load\_all}{Load all R scripts in subdirectories}{load.Rul.all}
\keyword{internal}{load\_all}
%
\begin{Description}
Load all R scripts in subdirectories
\end{Description}
\HeaderA{merge\_enrichment\_results}{Merge List of Enrichment Results}{merge.Rul.enrichment.Rul.results}
%
\begin{Description}
This function merges multiple enrichment results ('enrichment\_results') into a single dataframe by
combining unique GeneID elements across each geneset, and averaging Pvalue / Padj
values for each term across all enrichment\_results.
\end{Description}
%
\begin{Usage}
\begin{verbatim}
merge_enrichment_results(enrichment_results)
\end{verbatim}
\end{Usage}
%
\begin{Arguments}
\begin{ldescription}
\item[\code{enrichment\_results}] A list of geneset dataframes containing columns c('Term', 'GeneID', 'Pvalue', 'Padj')
\end{ldescription}
\end{Arguments}
%
\begin{Value}
A single merged geneset dataframe with all original columns
suffixed with the index of the geneset, with new columns 'GeneID', 'Pvalue',
'Padj' containing the merged values.
\end{Value}
\HeaderA{plot\_network\_graph}{Plot Network Graph for a Cluster}{plot.Rul.network.Rul.graph}
%
\begin{Description}
This function visualizes a single cluster as a network graph.
\end{Description}
%
\begin{Usage}
\begin{verbatim}
plot_network_graph(
  cluster_result,
  cluster_num,
  distance_matrix,
  valuetype_list
)
\end{verbatim}
\end{Usage}
%
\begin{Arguments}
\begin{ldescription}
\item[\code{cluster\_result}] The result from the clustering function.

\item[\code{cluster\_num}] The cluster number to plot.

\item[\code{distance\_matrix}] The distance matrix used for clustering.

\item[\code{valuetype\_list}] A list of value types (e.g., "Pvalue\_1", "Padj\_1") to use for node coloring.
\end{ldescription}
\end{Arguments}
%
\begin{Value}
A plot object.
\end{Value}
\HeaderA{runRichCluster}{Run clustering in C++ backend}{runRichCluster}
%
\begin{Description}
Run clustering in C++ backend
\end{Description}
%
\begin{Usage}
\begin{verbatim}
runRichCluster(
  terms,
  geneIDs,
  distanceMetric,
  distanceCutoff,
  linkageMethod,
  linkageCutoff
)
\end{verbatim}
\end{Usage}
%
\begin{Arguments}
\begin{ldescription}
\item[\code{terms}] Character vector of term names

\item[\code{geneIDs}] Character vector of geneIDs

\item[\code{distanceMetric}] e.g. "kappa"

\item[\code{distanceCutoff}] numeric between 0 and 1

\item[\code{linkageMethod}] e.g. "average"

\item[\code{linkageCutoff}] numeric between 0 and 1
\end{ldescription}
\end{Arguments}
\HeaderA{term\_bar}{Term-level Bar Plot for a Specific Cluster}{term.Rul.bar}
%
\begin{Description}
Creates a horizontal bar plot showing enrichment values for individual terms
in a selected cluster.
\end{Description}
%
\begin{Usage}
\begin{verbatim}
term_bar(cluster_result, cluster = 1, value_type = "Padj", title = NULL)
\end{verbatim}
\end{Usage}
%
\begin{Arguments}
\begin{ldescription}
\item[\code{cluster\_result}] A result list returned by \code{\LinkA{cluster}{cluster}}.

\item[\code{cluster}] Cluster ID (numeric) or term name (character) to visualize.

\item[\code{value\_type}] The column name to use for enrichment significance ("Padj" or "Pvalue").

\item[\code{title}] Optional plot title. If NULL, a default will be generated.
\end{ldescription}
\end{Arguments}
%
\begin{Value}
A \code{plotly} object representing the bar plot.
\end{Value}
%
\begin{Examples}
\begin{ExampleCode}
## Not run: 
tbar <- term_bar(cluster_result, cluster = 1)
tbar

## End(Not run)
\end{ExampleCode}
\end{Examples}
\HeaderA{term\_dot}{Term-level Dot Plot for a Specific Cluster}{term.Rul.dot}
%
\begin{Description}
Creates a dot plot of individual terms within a specified cluster, showing
their significance and number of genes.
\end{Description}
%
\begin{Usage}
\begin{verbatim}
term_dot(cluster_result, cluster = 1, value_type = "Padj", title = NULL)
\end{verbatim}
\end{Usage}
%
\begin{Arguments}
\begin{ldescription}
\item[\code{cluster\_result}] A result list returned from \code{\LinkA{cluster}{cluster}}.

\item[\code{cluster}] Cluster ID (numeric) or term name (character) to plot.

\item[\code{value\_type}] The name of the value column to visualize (e.g., "Padj" or "Pvalue").

\item[\code{title}] Optional title for the plot. If NULL, a default title is generated using the representative term.
\end{ldescription}
\end{Arguments}
%
\begin{Value}
A \code{plotly} object representing the dot plot of terms.
\end{Value}
%
\begin{Examples}
\begin{ExampleCode}
## Not run: 
tdot <- term_dot(cluster_result, cluster = 1)
tdot

## End(Not run)
\end{ExampleCode}
\end{Examples}
\HeaderA{term\_hmap}{Generate a Heatmap of Enrichment Results for Specific Clusters and Terms}{term.Rul.hmap}
%
\begin{Description}
Creates an interactive heatmap displaying -log10(Padj) values for selected clusters
and terms. Users can specify clusters numerically or select them by providing term names.
The function ensures that the final heatmap includes all terms from the selected clusters
as well as any explicitly provided terms.
\end{Description}
%
\begin{Usage}
\begin{verbatim}
term_hmap(cluster_result, clusters, terms, value_type, aggr_type, title = NULL)
\end{verbatim}
\end{Usage}
%
\begin{Arguments}
\begin{ldescription}
\item[\code{cluster\_result}] A list containing a data frame (`cluster\_df`) with clustering results.
The data frame must include at least the columns `Cluster`, `Term`, and `Padj\_*` values.

\item[\code{clusters}] Optional. A numeric vector specifying the cluster numbers to display,
or a character vector specifying terms whose clusters should be included. Defaults to `NULL`,
which includes all clusters.

\item[\code{terms}] Optional. A character vector specifying additional terms to include in the heatmap.
Defaults to `NULL`.

\item[\code{value\_type}] A character string specifying the column name prefix for adjusted p-values.
Defaults to `"Padj"`.

\item[\code{aggr\_type}] A function used to aggregate values across clusters (e.g., `mean` or `median`).
Defaults to `mean`.

\item[\code{title}] An optional parameter to title the plot something else.
\end{ldescription}
\end{Arguments}
%
\begin{Details}
The function processes the given `cluster\_df`, identifying the clusters and terms to be visualized.
If `clusters` is specified as a numeric vector, the function directly filters based on cluster numbers.
If `clusters` is given as a character vector, it identifies the clusters associated with those terms
and retrieves all terms from the selected clusters.

The `Padj\_*` values are transformed using `-log10()`, and infinite values are replaced with `0`.
The resulting heatmap is generated using `heatmaply::heatmaply()` with fixed row ordering
(no hierarchical clustering).
\end{Details}
%
\begin{Value}
An interactive heatmap object (`plotly`), displaying the -log10(Padj) values
across clusters, with representative terms as row labels and color-coded cluster annotations.
\end{Value}
\printindex{}
\end{document}
